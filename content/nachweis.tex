%%%%%%%%%%%%%%%%%%%%%%%%%%%%%%%%%%%%%%%%%%%%%%%%%%%%%%%%%%%%%%%%%
%_____________ ___    _____  __      __ 
%\____    /   |   \  /  _  \/  \    /  \  Institute of Applied
%  /     /    ~    \/  /_\  \   \/\/   /  Psychology
% /     /\    Y    /    |    \        /   Zürcher Hochschule 
%/_______ \___|_  /\____|__  /\__/\  /    fuer Angewandte Wissen.
%        \/     \/         \/      \/                           
%%%%%%%%%%%%%%%%%%%%%%%%%%%%%%%%%%%%%%%%%%%%%%%%%%%%%%%%%%%%%%%%%
%
% Project     : Latex Vorlage Nachweis
% Title       : 
% File        : einleitung.tex Rev. 00
% Date        : 24.10.2012
% Author      : Till J. Ernst
%
%%%%%%%%%%%%%%%%%%%%%%%%%%%%%%%%%%%%%%%%%%%%%%%%%%%%%%%%%%%%%%%%%
\chapter*{Nachweis - Der Traum}\label{chap.traum}
\glsresetall
\par
\begingroup
\leftskip=1cm
\rightskip=1.5cm
\noindent \textquotedblleft Träume sind keine beabsichtigten und willkürlichen Erfindungen, sondern natürlich Phänomene, die nichts anderes sind, als was sie eben darstellen. Sie täuschen nicht, sie lügen nicht, sie verdrehen und vertuschen nicht, sondern verkünden naiv das, was sie sind und meinen.\textquotedblright, \cite{Jung:1972}.
\par
\endgroup
\section*{Einleitung}\label{section.einleitung}
Dieser Nachweis befasst sich mit den Traumtheorien von C.G. Jung. In einem ersten theoretischen Teil werden wichtige Elemente von Jung in Form einer kurzen Zusammenfassung vorgestellt. Anschliessend folgt ein persönlicher Traum, der sich im Zeitraum der Vorlesung an der ZHAW ereignete. In der abschliessenden Diskussion wird versucht, einen persönlicher Bezug von diesem Traum zu den Jungschen Theorien herzustellen.
\section*{Theorie}\label{section.theorie}
Der Traum als “Königsweg” zum Unbewussten, als die “via regia” zu verdrängten und dadurch unbewussten Lebensthemen, als Erkenntnisquelle unbewusster Inhalte. All diese Aspekte wurden vom Gründervater der Psychoanalyse S. Freud in seiner Veröffentlichung der Traumdeutung \cite{Freud:1900} beschrieben. Gemäss \citeA{Roth:2011} war das die Geburtsstunde der Psychoanalyse. Jung kam mit diesem Werk an der Psychiatrischen Universitätsklinik von Zürich, im “Burghölzli”, in Kontakt. Im Gegensatz zu Freud, der die Bedeutung der Träume im Zuge seiner Arbeit immer mehr relativierte, behielt Jung die zentrale Bedeutung der Träume als Ausdruck unbewusster Prozesse bei (ebda, 2011). \newline
\citeA{Roth:2011} beschreibt in seiner Arbeit, dass für Jung der Traum die unmittelbare Symbolsprache des Unbewussten ist. Als symbolhafter Ausdruck unbewusster Inhalte, dem eine kreative und schöpferische Tiefe zugrunde liegt. Im Konzept gemäss Freud, entstehen Träume vornehmlich aus verdrängten, meist sexuellen, Triebregungen, die sich in verhüllter Form darstellen. Der erinnerbare Traum ist somit das Ergebnis einer triebhaften Aktivität, die vom Bewusstsein nicht zugelassen wurde und sich als manifester Traum zeigt (auch als verfälschter Traum bezeichnet). Der latente oder eigentliche Trauminhalt gilt es in der Freudschen-Analyse mittels freier Assoziation zu erschliessen \cite{Stuessi:2013}. Dem gegenüber sieht Jung den Traum als Symbolsprache des Unbewussten, in der individuelle und kollektive unbewusste Inhalte dargestellt werden. Die Funktion des Traums bei Freud hat hauptsächlich einen symptomatischen Wert. Er steht somit im Dienste eines reduktiv-kausalen Verständnissen \cite{Roth:2011}. Wohingegen Jung die prospektiv-finale Auffassung vertrat (ebda, 2011), die dem Traum einen entwicklungsorientierten und zielgerichteten Sinn zuschreibt. Die Symbolsprache des Traumes ist gemäss Jung die jeweils bestmögliche Ausdrucksform, um einen bestimmten Sachverhalt rational und emotional darzustellen. \newline
Gemäss Freud dienen Träume dazu, vergangene, verdrängte und damit unbewusste Aspekte dem Unbewussten zugänglich zu machen \cite{Roth:2011}. Durch die Darbietung unbewusster Inhalte sollen nicht bewältigte Lebensthemen und -konflikte erkannt und bearbeitet werden. Jung hingegen schrieb den Träumen eigenständige Fähigkeiten zu, die kreativ und konstruktiv auf die psychische Entwicklung und damit auf die Individuation im Sinne von Jung einwirken (ebda, 2011). Basierend auf dem Konzept des kollektiven Unbewussten, dem auch eine kreative und konstruktive Funktion zugeschrieben wird, ist der Traum einerseits eine Erkenntnisquelle, die es zu entschlüsseln gilt und andererseits auch ein Vermittler zwischen Unbewussten und Bewussten (Gegensatzpaar). Der Individuationsprozess basiert gemäss \citeA{Frey:1977} aus dem Resultat einer Auseinandersetzung zwischen Bewusstsein und Unbewusstem.\newline
Ein wesentlicher Faktor für ein erfolgreiches Arbeiten mit Träumen ist es, mit den symbolischen Bildern emotional in Kontakt zu treten \cite{Roth:2011}. Die Grundsätzliche Sichtweise bei der Traumdeutung ist sinnvermittelnd und damit kreativ-konstruktiv zu verstehen, was im Sinne des Individuationsgedankens steht. Der Traum entspricht einer Selbstverwirklichung, die sich auf eine prospektiv-finales Ziel fokussiert. Archetypische Träume gelten als Träger dieses Selbstentfaltungsprozesses (ebda, 2011).
\section*{Persönlicher Trauminhalt}\label{section.eigenerTraum}
In einem einleitenden Traum ging es um restliche Tagesreste, worauf ich hier aus Platzgründen nicht näher eingehe möchte. Dieser Traum verblasste und leitete in eine weitere Traumsequenz über: \par 
\textit{\textquotedblleft Das Umfeld begann sich zu ändern. Ich befand mich neuerdings in einem Raum, der wie dem Balkon in einem Theater glich. Die Bühne, die sich im Zentrum des Raumes befand, war um einen Stock nach unten versetzt. Zu dieser Bühne führte eine breite Treppe, die sich unmittelbar vor mir befand. Der untere Teil war von meinem Standort mittels Geländer getrennt. Nur da wo die Treppe hinunter führte, war das Geländer unterbrochen. Ein innerer drang bewog mich, die Treppe hinunter zu steigen. Der untere Teil hatte die Form eines Kreises. Er sah wie die Bühne eines alten Theaters aus und hatte einen Boden aus Holz. Am Ende der Bühne befand sich etwas, das wie ein zugezogener Vorhang ausschaute. Auf dem Weg die Treppe hinunter hatte ich die Gewissheit, dass ich auf dem Weg war, eine wichtige Entscheidung zu treffen. Dies wurde mir klarer, je näher ich mich dem Ende der Treppe näherte. \newline 
Unten befanden sich verschiedene Personen, die ich nicht genau erkennen konnte. Ich wusste aber, dass sie alle einen spezifischen Charakter spielten. Meine Aufgabe war es, mich für einen Charakter zu entscheiden und ihn für mich auszulesen. Dieser Charakter würde mich vertreten und sollte dann, ähnlich wie bei einem Gladiatorkampf, für mich kämpfen. Ich spürte, dass diese Entscheidung sehr wichtig für mich war und dass ich mich weise entscheiden musste. Ich hatte keine Angst vor der Entscheidung. Im Gegenteil. Ich war sehr ruhig und gelassen, obwohl von diesem Entscheid meine Zukunft abhing. Die ganze Situation war nicht ganz neu für mich. Ich wusste, dass ich mich zu einem früheren Zeitpunkt bereits einmal für einen Charakter entschieden hatte. \newline 
Die Zeit war reif, für eine neue Wahl. Mir viel ein schon etwas älterer Mann auf, der sich etwas zurückhaltend, mit verschränkten Armen, zu meiner linken Seite befand. Dieser Mann hatte etwas stolzes an sich. Seine Haare waren kurz geschnitten und bereits etwas ergraut. Seine Postur war kräftig und athletisch. Obwohl er nur so dastand, drückte er etwas faszinierendes aus. Ich entschied mich für ihn, weil er mich durch sein Wesen ansprach und ich mich vertraut zu ihm hingezogen fühlte. Ich sprach ihn bei meiner Wahl mit ‘Du’ an. Darauf  erwiderte er an niemanden besonders gerichtet, dass er noch nie mit mir zusammen Schweine gehütet hätte. Wohl als Reaktion auf meine unflätige Art, ihn direkt angesprochen zu haben. Ich antwortet mit einem Lächeln an meinen Begleiter gerichtet, dass die widerspenstigen Menschen doch immer die Besten seien. Mein Begleiter stand bereits die ganze zeit neben mir, ich bemerkte in aber erst nachdem ich mich mit meinem zukünftigen Charakter auseinander gesetzt hatte. Als der Mann meine Bemerkung hörte, huschte ein Lächeln über sein Gesicht. Ich hatte das Gefühl, dass er sich im Grunde über meine Wahl freute. Ich merkte, dass wir uns gut verstehen würden. \newline 
Da ich mich aber für diesen Charakter entschieden hatte, war mein alter Charakter nicht mehr nötig. Dieser wurde von einem jungen Mann verkörpert, der mich stark an einen amerikanischen Sunnyboy erinnerte. Er hatte glattes, glänzendes Haar mit Dauerwellen. Sein Oberkörper war zum teil entblösst, so dass seine trainierten Muskeln zu sehen waren. Er hatte braungebrannte Haut, die leicht schimmerte. Er drückte etwas Überhebliches aus. Als er sich abwandte blickte er spöttisch über seine Schulter und kniff dabei den Mund zu einem Grinsen zusammen. Er murmelte noch etwas Unverständliches im Sinne von, dass er nicht auf mich angewiesen sei und ich ihn irgendwann zurückwolle. Danach lief er davon.\textquotedblright}
\section*{Diskussion}\label{section.diskussion}




